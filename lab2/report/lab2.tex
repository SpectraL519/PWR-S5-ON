\documentclass[12pt]{article}
\usepackage[margin=2cm]{geometry}
\usepackage{titling}
\usepackage[T1]{fontenc}
\usepackage{tabularx}
\usepackage{amsfonts}
\usepackage{amsmath}
\usepackage{graphicx}
\usepackage{changepage}


\pretitle{\begin{center}\Huge\bfseries}
\posttitle{\par\end{center}\vskip 0.5em}
\preauthor{\begin{center}\Large}
\postauthor{\end{center}}
\predate{\par\large\centering}
\postdate{\par}

\title{Obliczenia naukowe - lab 2}
\author{Jakub Musiał 268442}
\date{Listopad 2023}

\begin{document}

\maketitle

\hspace{1cm}

\section*{Zadanie 1 - Iloczyn skalarny}
    \subsection*{Problem}
        Powtórzyć experyment z zadania 5 z listy 1 dla zmienionych danych.
        \newline
        Experyment - porównanie metod sumowania iloczynów w algorytmie obliczania iloczynu skalarnego (w przód, w tył, od największego do najmniejszego oraz od najmniejszego do najwięszego) w pojedynczej oraz podwójnej precyzji.
        \newline

        \noindent Oryginalne wartości wektorów:
            \begin{center}
            \begin{math}
                \left\{
                    \begin{array}{l}
                        x = [2.718281828, -3.141592654, 1.414213562, 0.5772156649, 0.3010299957] \\
                        y = [1486.2497, 878366.9879, -22.37492, 4773714.647, 0.000185049]
                    \end{array}
                \right.
            \end{math}
            \end{center}

        \noindent \newline
        W tym eksperymencie modyfikacja danych to usunięcie ostatniej $9$ z $x_4$ oraz ostatniej $7$ z $x_5$.
        \newline

    \subsection*{Rozwiązanie}
        Program z rozwiązaniem: \texttt{ex1.jl}
        \newline

    \subsection*{Wyniki i obserwacje}
        W \textit{tabelach 1 i 2} możemy zauważyć, że obliczone wartości dla pojedynczej precyzji są identyczne, co wynika z tego, że obcieliśmy wartości na $10$ miejscu po przecinku, co przekracza zakres precyzji - $\varepsilon_{float32} = 2^{-24} \approx 6 \cdot 10^{-8}$.
        \newline
        Dla precyzji podwójnej (\textit{tabele 3 i 4}) jednak możemy zaobserwować duże zmiany rzędów uzyskanych wartości - dla metod sumowania "w przód" i "w tył" jest to zmiana o $10^{10}$.
        \newline
        To pokazuje, że, rozważając iloczyn skalarny wektorów prawie prostopadłych, bardzo małe modyfikacje danych wejściowych powodują duże zmiany wyników.
        \newline
        Zadanie jest więc źle uwarunkowane.

        \newpage

        \begin{table}[h!]
            \centering
            \begin{tabularx}{0.67\textwidth}{l l l}
                \hline
                Algorytm & $S$ & $\delta$ \\
                \hline
                W przód & $-0.4999443$ & $4.966805766328285e10$ \\
                W tył & $-0.4543457$ & $4.5137965582009605e10$ \\
                Od najwięszych & $-0.5$ & $4.967359135506108e10$ \\
                Od najmniejszych & $-0.5$ & $4.967359135506108e10$ \\
                \hline
            \end{tabularx}
            \caption{Wartości iloczynu skalarnego wraz z błędem względnym w precyzji \texttt{Float32} (dane z zadania 5 z listy 1)}
            \label{table:dot_prod_32}
        \end{table}

        \begin{table}[h!]
            \vspace{20pt}
            \centering
            \begin{tabularx}{0.67\textwidth}{l l l}
                \hline
                Algorytm & $S$ & $\delta$ \\
                \hline
                W przód & $-0.4999443$ & $4.966805766328285e10$ \\
                W tył & $-0.4543457$ & $4.5137965582009605e10$ \\
                Od najwięszych & $-0.5$ & $4.967359135506108e10$ \\
                Od najmniejszych & $-0.5$ & $4.967359135506108e10$ \\
                \hline
            \end{tabularx}
            \caption{Wartości iloczynu skalarnego wraz z błędem względnym w precyzji \texttt{Float32} dla zmienionych danych}
            \label{table:dot_prod_32_mod}
        \end{table}

        \begin{table}[h!]
            \vspace{20pt}
            \centering
            \begin{tabularx}{0.78\textwidth}{l l l}
                \hline
                Algorytm & $S$ & $\delta$ \\
                \hline
                W przód & $1.0251881368296672e-10$ & $9.184955313981629$ \\
                W tył & $-1.5643308870494366e-10$ & $16.54118664516592$ \\
                Od najwięszych & $0.0$ & $1.0$ \\
                Od najmniejszych & $0.0$ & $1.0$ \\
                \hline
            \end{tabularx}
            \caption{Wartości iloczynu skalarnego wraz z błędem względnym w precyzji \texttt{Float64} (dane z zadania 5 z listy 1)}
            \label{table:dot_prod_64}
        \end{table}

        \begin{table}[h!]
            \vspace{20pt}
            \centering
            \begin{tabularx}{0.78\textwidth}{l l l}
                \hline
                Algorytm & $S$ & $\delta$ \\
                \hline
                W przód & $-0.004296342739891585$ & $4.2682954815672344e8$ \\
                W tył & $-0.004296342998713953$ & $4.2682957386999655e8$ \\
                Od najwięszych & $-0.004296342842280865$ & $4.268295583288099e8$ \\
                Od najmniejszych & $-0.004296342842280865$ & $4.268295583288099e8$ \\
                \hline
            \end{tabularx}
            \caption{Wartości iloczynu skalarnego wraz z błędem względnym w precyzji \texttt{Float64} dla zmienionych}
            \label{table:dot_prod_64_mod}
        \end{table}

\newpage

\section*{Zadanie 2 - Badanie granicy funkcji}
    \subsection*{Problem}
        Obliczyć granicę funkcji $f(x) = e^x \cdot \ln(1 + e^{-x})$ i porównać uzyskany wynik z wygenerowanymi w 2 programach do wizualizacji wykresami zadanej funkcji.
        \newline

    \subsection*{Rozwiązanie}
        Policzmy granicę funkcji $f$ w nieskończoności:

        \begin{center}
        \begin{math}
            \lim_{x \rightarrow \infty} e^x \cdot \ln(1 + e^{-x}) = \lim_{x \rightarrow \infty} \frac{\ln (1+e^{-x})}{e^{-x}} \overset{\text{de l'Hospital}}{=} \lim_{x \rightarrow \infty} \frac{-\frac{e^{-x}}{1+e^{-x}}}{-e^{-x}} = \lim_{x \rightarrow \infty} \frac{1}{1+e^{-x}} = 1
        \end{math}
        \end{center}

        \noindent \newline Wizualizacje wykresów funkcji $f$ wykonałem w programach:
        \begin{itemize}
            \item \texttt{Julia} - biblioteka \texttt{Plots} (\texttt{ex2.jl})
            \item \texttt{Python} - biblioteka \texttt{matplotlib.pyplot} (\texttt{ex2.py})
        \end{itemize}

        \noindent \newline

    \subsection*{Wyniki i obserwacje}
        Na poniższych wykresach można zauważyć rozbieżność z wyliczoną wcześniej granicą funkcji. Zauważmy, że dla $x < x_0$ bliskich $x_0$ wartości funkcji są coraz bardziej rozbieżne oraz $(\forall x > x_0)(f(x) = 0)$.
        \newline\newline
        To wynika z ograniczonej prezycji arytmetyki zmiennopozycyjnej. W podwójnej precyzji standardu \texttt{IEEE 754} mamy $52$ bity na reprezentację mantysy.
        \newline\newline
        Dla $e^{-x} < 2^{-52} \implies x > 52 \cdot \ln(2) \approx 36.04365 = x_0$ mamy więc:
        \begin{center}
        \begin{math}
            fl(1 + e^{-x}) = 1 \\
            \Downarrow \\
            fl(\ln(1 + e^{-x})) = fl(ln(1)) = 0 \\
            \Downarrow \\
            fl(f(x)) = fl(e^x \cdot \ln(1 + e^{-x})) = fl(e^x \cdot 0) = 0
        \end{math}
        \end{center}

        \newpage

        \begin{figure}[h!]
            \centering
            \includegraphics[width=0.7\linewidth, clip]{img/ex2_jl.png}
            \caption{Wykres funkcji $f$ uzyskany w języku \texttt{Julia}}
            \label{fig:ex2_jl}
        \end{figure}

        \begin{figure}[h!]
            \centering
            \includegraphics[width=0.7\linewidth, clip]{img/ex2_py.png}
            \caption{Wykres funkcji $f$ uzyskany w języku \texttt{Python}}
            \label{fig:ex2_jl}
        \end{figure}

\newpage

\section*{Zadanie 3 - Rozwiązywanie liniowych układów rownań}
    \subsection*{Problem}
        Rozwiązać układ równań liniowych $Ax = b$, gdzie $A \in \mathbb{R}^{n \times n}$ jest wygenerowaną macierzą współczynników, a $b = Ax : x = (1, ..., 1)^T$ jest wektorem prawych stron.
        \newline\newline
        Zadany układ równań należy rozwiązać metodą eliminacji Gaussa oraz metodą inwersji dla macierzy $A$ generowanej jako:
        \begin{itemize}
            \item $A = \mathbf{H}_n$ - macierz Hilberta n-tego stopnia
            \item $A = \mathbf{R}_n$ - losowa macierz kwadratowa stopnia n (dla ustalonych wartości $cond(A)$)
        \end{itemize}

    \subsection*{Rozwiązanie}
        Program z rozwiązaniem: \texttt{ex3.jl}

    \subsection*{Wyniki i obserwacje}
        Na podstawie poniższych tabel możemy zauważyć, że obie metody dają wyniki o zbliżonej dokładności zarówno dla macierzy Hilberta, jak i macierzy losowych.
        \newline
        Dodatkowo, patrząc na wyniki dla macierzy losowych, możemy zauważyć, że rozmiar macierzy ma niewielkie znaczenie dla błędów badanych metod - dla macierzy o różnych rozmiarach, ale takich samych wskaźnikach uwarunkowania uzyskujemy wyniki o bardzo zbliżonych błędach.
        \newline
        Porównując także wyniki uzyskane dla losowych macierzy z tymi dla macierzy Hilberta, lecz dla podobnych wartości $cond(A)$ - przykładowo porównując wyniki dla macierzy $\mathbf{H_6}$ oraz $A = \mathbf{R_5} : cond(A) = 10^7$ -  również zaobserwujemy zbliżone rzędy błedów.
        \newline
        Możemy zatem stwierdzić, że głównym czynnikiem determinującym rząd wielkości błędu rozwiązania jest wskaźnik uwarunkowania, co znaczy, że algorytmy działają poprawnie, natomiast zadania mogą być źle uwarunkowane, jak w przypadku macierzy Hilberta.
        \newline

        \newpage

        \begin{table}[h!]
            \begin{tabularx}{\textwidth}{l l l l l }
                \hline
                $n$ & $cond(A)$ & $rank(A)$ & $\delta_{gauss}$ & $\delta_{inversion}$ \\
                \hline
                $1$ & $1.0$ & $1$ & $0.0$ & $0.0$ \\
                $2$ & $19.281470067903967$ & $2$ & $5.661048867003676e-16$ & $1.1240151438116956e-15$ \\
                $3$ & $524.0567775860627$ & $3$ & $8.351061872731819e-15$ & $9.825526038180824e-15$ \\
                $4$ & $15513.738738929662$ & $4$ & $4.2267316576255873e-13$ & $3.9600008750140806e-13$ \\
                $5$ & $476607.2502419338$ & $5$ & $1.256825919192874e-12$ & $8.128168770215688e-12$ \\
                $6$ & $1.4951058641781438e7$ & $6$ & $1.5435074657413347e-10$ & $1.0423794065751672e-10$ \\
                $7$ & $4.753673568815896e8$ & $7$ & $6.520804933066021e-9$ & $4.3299229851434615e-9$ \\
                $8$ & $1.5257575568347378e10$ & $8$ & $3.6010489197068436e-7$ & $4.0236799996435915e-7$ \\
                $9$ & $4.9315332284138226e11$ & $9$ & $1.3216991540025553e-5$ & $1.4626798972086921e-5$ \\
                $10$ & $1.602493053861801e13$ & $10$ & $0.0004194170177181955$ & $0.00040714905218460087$ \\
                $11$ & $5.2247797042670644e14$ & $10$ & $0.01004906783345069$ & $0.010645959401385671$ \\
                $12$ & $1.642582584316724e16$ & $11$ & $0.5502106922296848$ & $0.6697890564301745$ \\
                $13$ & $4.4691072287967375e18$ & $11$ & $70.1556197115221$ & $82.66675811171989$ \\
                $14$ & $3.213787720967528e17$ & $11$ & $9.649642437452474$ & $10.094732062453225$ \\
                $15$ & $3.3659212797388026e17$ & $12$ & $692.4295360390742$ & $715.740988667373$ \\
                $16$ & $2.2179227009259395e18$ & $12$ & $10.414656083840297$ & $8.442143351389534$ \\
                $17$ & $9.830888596045286e17$ & $12$ & $18.67581817300634$ & $17.157982115668773$ \\
                $18$ & $2.58632622754956e18$ & $12$ & $5.40548300394664$ & $3.742412802776696$ \\
                $19$ & $3.42284739358336e18$ & $13$ & $15.073941146224387$ & $16.84769281513296$ \\
                $20$ & $6.806966466008104e18$ & $13$ & $28.79267493699834$ & $30.751202239608727$ \\
                \hline
            \end{tabularx}
            \label{table:hilbert_matrix_error}
            \caption{Błędy względne wyników obliczonych metodą Gaussa i inversji dla macierzy Hilberta}
        \end{table}

        \newpage

        \begin{table}
            \begin{tabularx}{\textwidth}{l l l l l }
                \hline
                $n$ & $cond(A)$ & $rank(A)$ & $\delta_{gauss}$ & $\delta_{inversion}$ \\
                \hline
                $5$ & $10^{0}$ & $5$ & $1.719950113979703e-16$ & $1.7901808365247238e-16$ \\
                $5$ & $10^{1}$ & $5$ & $1.2161883888976234e-16$ & $1.719950113979703e-16$ \\
                $5$ & $10^{3}$ & $5$ & $4.606673749484534e-14$ & $5.32315529767373e-14$ \\
                $5$ & $10^{7}$ & $5$ & $4.6258744328352526e-10$ & $4.428329574522588e-10$ \\
                $5$ & $10^{12}$ & $5$ & $1.4311845989993493e-5$ & $1.3892648149843546e-5$ \\
                $5$ & $10^{16}$ & $4$ & $0.4701807484683632$ & $0.5764835175254149$ \\
                $10$ & $10^{0}$ & $10$ & $3.2368285245694683e-16$ & $1.7554167342883504e-16$ \\
                $10$ & $10^{1}$ & $10$ & $5.404859424301949e-16$ & $4.657646926676368e-16$ \\
                $10$ & $10^{3}$ & $10$ & $1.0871629027882736e-14$ & $1.3286081194048358e-14$ \\
                $10$ & $10^{7}$ & $10$ & $5.3124054213478024e-11$ & $1.348469495294427e-10$ \\
                $10$ & $10^{12}$ & $10$ & $2.452555006986748e-5$ & $2.0626679921248615e-5$ \\
                $10$ & $10^{16}$ & $9$ & $0.19643775452930137$ & $0.18250673926399288$ \\
                $20$ & $10^{0}$ & $20$ & $6.030054506389723e-16$ & $4.742874840267547e-16$ \\
                $20$ & $10^{1}$ & $20$ & $5.461575137144e-16$ & $2.7081250392525437e-16$ \\
                $20$ & $10^{3}$ & $20$ & $1.1811750363909761e-14$ & $1.1291792337224575e-14$ \\
                $20$ & $10^{7}$ & $20$ & $1.0828552075050355e-12$ & $8.5128898422199e-11$ \\
                $20$ & $10^{12}$ & $20$ & $1.7043123271689405e-5$ & $1.764361921628692e-5$ \\
                $20$ & $10^{16}$ & $19$ & $0.24777843015168452$ & $0.199723624960062$ \\
                \hline
            \end{tabularx}
            \label{table:random_matrix_error}
            \caption{Błędy względne wyników obliczonych metodą Gaussa i inversji dla macierzy losowych}
        \end{table}
        \noindent\newline

\newpage

\section*{Zadanie 4 - Pierwiastki wielomianu Wilkinsona}
    \subsection*{Problem}
        Obliczyć pierwiastki wielomianu Wilkinsona $p(x) = \Pi_{i=1}^{20} (x - i)$ za pomocąfunkcji \texttt{roots} z pakietu \texttt{Polynomials} języka \texttt{Julia}, znając postać normalną $P$ tego wielomianu oraz porównać uzyskane wyniki z jego rzeczywistymi pierwiastkami.
        \newline
        Następnie należy powtórzyć eksperyment dla zmodyfikowanego współczynnika przy $x^{19}$ ($-210 - 2^{-23}$ zamiast $-210$).

    \subsection*{Rozwiązanie}
        Program z rozwiązaniem: \texttt{ex4.jl}

    \subsection*{Wyniki i obserwacje}
        W \textit{tabeli 7} możemy zauważyć, że obliczone pierwiastki wielomianu są bardzo zbliżone do rzeczywistych wartości, jednak ze względu na to, że w postaci normalnej współczynniki wielomianu są bardzo duże dla niższych potęch $x$, dostajemy wyniki bardzo oddalone od oczekiwanego - $0$.
        \newline
        Dodatkowo w \textit{tabeli 8} możemy zauważyć, że niewielka zmiana wartości jednego ze współczynników powoduje, że część pierwiastków nie jest możliwa do obliczenia w ciele $\mathbb{R}$ i przez to otrzymujemy wyniki zespolone.
        \newline
        Te dwa spostrzeżenia pokazują, że zadanie jest źle uwarunkowane.

        \begin{table}[h!]
            \begin{adjustwidth}{-1cm}{}
            \begin{tabularx}{1.1\textwidth}{l l l l l }
                \hline
                $k$ & $z_k$ & $|P(z_k)|$ & $|p(z_k)|$ & $|z_k - k|$ \\
                \hline
                $1$ & $0.9999999999998084$ & $23323.616390897252$ & $24347.616390897056$ & $1.9162449405030202e-13$ \\
                $2$ & $2.0000000000114264$ & $64613.550791712885$ & $80997.5507918065$ & $1.1426415369442111e-11$ \\
                $3$ & $2.9999999998168487$ & $18851.098984644806$ & $101795.09897958105$ & $1.8315127192636282e-10$ \\
                $4$ & $3.999999983818672$ & $2.6359390809003003e6$ & $2.37379508196076e6$ & $1.6181327833209025e-8$ \\
                $5$ & $5.000000688670983$ & $2.3709842874839526e7$ & $2.306984278668964e7$ & $6.88670983350903e-7$ \\
                $6$ & $5.999988371602095$ & $1.2641076289358065e8$ & $1.2508366146559621e8$ & $1.162839790502801e-5$ \\
                $7$ & $7.000112910766231$ & $5.2301629899144447e8$ & $5.2055763533357024e8$ & $0.00011291076623098917$ \\
                $8$ & $7.999279406281878$ & $1.798432141726085e9$ & $1.7942387274786062e9$ & $0.0007205937181220534$ \\
                $9$ & $9.003273831140774$ & $5.121881552672067e9$ & $5.115158339932115e9$ & $0.003273831140774064$ \\
                $10$ & $9.989265687778465$ & $1.4157542666785017e10$ & $1.4147337313301882e10$ & $0.010734312221535092$ \\
                $11$ & $11.027997558569794$ & $3.586354765112257e10$ & $3.584844758752149e10$ & $0.027997558569794023$ \\
                $12$ & $11.94827395840048$ & $8.510931555828575e10$ & $8.508835580052173e10$ & $0.051726041599520656$ \\
                $13$ & $13.082031971969954$ & $2.2136146301419052e11$ & $2.2133136429365445e11$ & $0.08203197196995404$ \\
                $14$ & $13.906800565193148$ & $3.812024574451268e11$ & $3.811638891032201e11$ & $0.09319943480685211$ \\
                $15$ & $15.081439299377482$ & $8.809029239560208e11$ & $8.808498064028993e11$ & $0.0814392993774824$ \\
                $16$ & $15.942404318674466$ & $1.6747434633806333e12$ & $1.6746775852847332e12$ & $0.05759568132553383$ \\
                $17$ & $17.026861831476396$ & $3.3067827086376123e12$ & $3.3066967643946914e12$ & $0.026861831476395537$ \\
                $18$ & $17.99048462339055$ & $6.166202940769282e12$ & $6.16609562105193e12$ & $0.009515376609449788$ \\
                $19$ & $19.001981084996206$ & $1.406783619602919e13$ & $1.4067702719729383e13$ & $0.001981084996206306$ \\
                $20$ & $19.999803908064397$ & $3.284992217648231e13$ & $3.2849758339688676e13$ & $0.00019609193560299332$ \\
                \hline
            \end{tabularx}
            \label{table:Wilkinson}
            \caption{Pierwiastki wielomianiu Wilkinsona}
            \end{adjustwidth}
        \end{table}

        \newpage

        \begin{table}[h!]
            \begin{adjustwidth}{-0.8cm}{}
            \begin{tabularx}{1.065\textwidth}{l l l l }
                \hline
                $k$ & $z_k$ & $|P'(z_k)|$ & $|z_k - k|$ \\
                \hline
                $1$ & $0.9999999999999805 + 0.0i$ & $2168.9361669986724$ & $1.9539925233402755e-14$ \\
                $2$ & $1.9999999999985736 + 0.0i$ & $29948.438957395843$ & $1.4264145420384011e-12$ \\
                $3$ & $3.000000000105087 + 0.0i$ & $239010.53520956426$ & $1.0508705017286957e-10$ \\
                $4$ & $3.9999999950066143 + 0.0i$ & $939293.8049425513$ & $4.993385704921138e-9$ \\
                $5$ & $5.000000034712704 + 0.0i$ & $7.44868039679552e6$ & $3.4712703822492585e-8$ \\
                $6$ & $6.000005852511414 + 0.0i$ & $1.4689332508961653e7$ & $5.852511414161654e-6$ \\
                $7$ & $6.999704466216799 + 0.0i$ & $5.817946400915084e7$ & $0.00029553378320112955$ \\
                $8$ & $8.007226654064777 + 0.0i$ & $1.3954205929609105e8$ & $0.0072266540647767386$ \\
                $9$ & $8.917396943382494 + 0.0i$ & $2.459617755654851e8$ & $0.082603056617506$ \\
                $10$ & $10.09529034477879 - 0.6432770896263527i$ & $2.291018560461982e9$ & $0.6502965968281023$ \\
                $11$ & $10.09529034477879 + 0.6432770896263527i$ & $2.291018560461982e9$ & $1.110092326920887$ \\
                $12$ & $11.793588728372308 - 1.6522535463872843i$ & $2.077690789102519e10$ & $1.6650968123818863$ \\
                $13$ & $11.793588728372308 + 1.6522535463872843i$ & $2.077690789102519e10$ & $2.0458176697496047$ \\
                $14$ & $13.99233053734825 - 2.5188196443048287i$ & $9.390730597798799e10$ & $2.5188313205122075$ \\
                $15$ & $13.99233053734825 + 2.5188196443048287i$ & $9.390730597798799e10$ & $2.7129043747424584$ \\
                $16$ & $16.73073008036981 - 2.8126272986972136i$ & $9.592356563898315e11$ & $2.906000476898456$ \\
                $17$ & $16.73073008036981 + 2.8126272986972136i$ & $9.592356563898315e11$ & $2.8254873227453055$ \\
                $18$ & $19.50243895868367 - 1.9403320231930836i$ & $5.050467401799687e12$ & $2.4540193937292005$ \\
                $19$ & $19.50243895868367 + 1.9403320231930836i$ & $5.050467401799687e12$ & $2.004328632592893$ \\
                $20$ & $20.84690887410499 + 0.0i$ & $4.858653129933677e12$ & $0.8469088741049902$ \\
                \hline
            \end{tabularx}
            \label{table:wilkinson_distorted}
            \caption{Pierwiastki zaburzonego wielomianu wilkinsona}
            \end{adjustwidth}
        \end{table}

\newpage

\section*{Zadanie 5 - Model wzrostu populacji}
    \subsection*{Problem}
        Wykonać $40$ iteracji wzoru opisującego model wzrostu populacji:
        \begin{center}
        \begin{math}
            p_{n+1} = p_n + r \cdot p_n \cdot (1 - p_n)
        \end{math}
        \end{center}
        \noindent Dla zadanych wartości: $p_0 = 0.01 \land r = 3$ w pojedynczej oraz podwójnej precyzji, a także w pojedynczej precyzji, obcinając wynik $p_10$ do $3$ miejsc po przecinku.

    \subsection*{Rozwiązanie}
        Program z rozwiązaniem: \texttt{ex5.jl}

    \subsection*{Wyniki i obserwacje}
        Na podstawie wyników z \textit{tabeli 9} możemy stwierdzić, że błedy obliczeniowe bardzo łatwo propagują się na dalsze obliczenia.
        \newline
        Porównując wyniki dla pojedynczej i podwójnej precyzji, możemy zauważyć, że od pewnego momentu niedokładność obliczeń wynika ze zbyt małej precyzji - nie jesteśmy w stanie dokładnie reprezentować kolejnych wartości, co wpływa na narastanie błędów w dalszych obliczeniach.
        \newline
        Podobną rzecz możemy zauważyć, porównując wyniki dla precyzji \texttt{Float32} z oraz bez obcięcia wyniku w $10$-tej iteracji. Tutaj błąd obliczeniowy staje się bardziej znaczący nawet szybciej niż w przypadku różnicy precyzji, którą wcześniej obserwowaliśmy, a ostateczny wynik jest zdecydowanie mniej dokładny.
        \newline
        Możemy zatem stwierdzić, że model ten numerycznie niestabilny, ponieważ niewielkie błedy bardzo łatwo propagują się na następne iteracje, skutkując niedokładnymi wynikami.

        \newpage

        \begin{table}[h!]
            \centering
            \begin{tabularx}{0.775\textwidth}{l l l l}
            \hline
            $n$ & $p_n$ - \texttt{Float32} & $p_n$ - \texttt{Float32} (obcięcie $p_{10}$) & $p_n$ - \texttt{Float64} \\
            \hline
            $0$ & $0.01$ & $0.01$ & $0.01$ \\
            $1$ & $0.0397$ & $0.0397$ & $0.0397$ \\
            $2$ & $0.15407173$ & $0.15407173$ & $0.15407173000000002$ \\
            $3$ & $0.5450726$ & $0.5450726$ & $0.5450726260444213$ \\
            $4$ & $1.2889781$ & $1.2889781$ & $1.2889780011888006$ \\
            $5$ & $0.1715188$ & $0.1715188$ & $0.17151914210917552$ \\
            $6$ & $0.5978191$ & $0.5978191$ & $0.5978201201070994$ \\
            $7$ & $1.3191134$ & $1.3191134$ & $1.3191137924137974$ \\
            $8$ & $0.056273222$ & $0.056273222$ & $0.056271577646256565$ \\
            $9$ & $0.21559286$ & $0.21559286$ & $0.21558683923263022$ \\
            $10$ & $0.7229306$ & $\mathbf{0.722}$ & $0.722914301179573$ \\
            $11$ & $1.3238364$ & $1.3241479$ & $1.3238419441684408$ \\
            $12$ & $0.037716985$ & $0.036488414$ & $0.03769529725473175$ \\
            $13$ & $0.14660022$ & $0.14195944$ & $0.14651838271355924$ \\
            $14$ & $0.521926$ & $0.50738037$ & $0.521670621435246$ \\
            $15$ & $1.2704837$ & $1.2572169$ & $1.2702617739350768$ \\
            $16$ & $0.2395482$ & $0.28708452$ & $0.24035217277824272$ \\
            $17$ & $0.7860428$ & $0.9010855$ & $0.7881011902353041$ \\
            $18$ & $1.2905813$ & $1.1684768$ & $1.2890943027903075$ \\
            $19$ & $0.16552472$ & $0.577893$ & $0.17108484670194324$ \\
            $20$ & $0.5799036$ & $1.3096911$ & $0.5965293124946907$ \\
            $21$ & $1.3107498$ & $0.09289217$ & $1.3185755879825978$ \\
            $22$ & $0.088804245$ & $0.34568182$ & $0.058377608259430724$ \\
            $23$ & $0.3315584$ & $1.0242395$ & $0.22328659759944824$ \\
            $24$ & $0.9964407$ & $0.94975823$ & $0.7435756763951792$ \\
            $25$ & $1.0070806$ & $1.0929108$ & $1.315588346001072$ \\
            $26$ & $0.9856885$ & $0.7882812$ & $0.07003529560277899$ \\
            $27$ & $1.0280086$ & $1.2889631$ & $0.26542635452061003$ \\
            $28$ & $0.9416294$ & $0.17157483$ & $0.8503519690601384$ \\
            $29$ & $1.1065198$ & $0.59798557$ & $1.2321124623871897$ \\
            $30$ & $0.7529209$ & $1.3191822$ & $0.37414648963928676$ \\
            $31$ & $1.3110139$ & $0.05600393$ & $1.0766291714289444$ \\
            $32$ & $0.0877831$ & $0.21460639$ & $0.8291255674004515$ \\
            $33$ & $0.3280148$ & $0.7202578$ & $1.2541546500504441$ \\
            $34$ & $0.9892781$ & $1.3247173$ & $0.29790694147232066$ \\
            $35$ & $1.021099$ & $0.034241438$ & $0.9253821285571046$ \\
            $36$ & $0.95646656$ & $0.13344833$ & $1.1325322626697856$ \\
            $37$ & $1.0813814$ & $0.48036796$ & $0.6822410727153098$ \\
            $38$ & $0.81736827$ & $1.2292118$ & $1.3326056469620293$ \\
            $39$ & $1.2652004$ & $0.3839622$ & $0.0029091569028512065$ \\
            $40$ & $0.25860548$ & $1.093568$ & $0.011611238029748606$ \\
            \hline
            \end{tabularx}
            \label{table:pupolation_growth}
            \caption{Wartości wyników wzoru rekurencyjnego dla poszczególnych iteracji}
        \end{table}

\newpage

\section*{Zadanie 6 - Ciąg rekurencyjny}
    \subsection*{Problem}
        Wykonać $40$ iteracji wzoru rekurencyjnego:
        \begin{center}
        \begin{math}
            x_{n+1} = x_n^2 + c
        \end{math}
        \end{center}
        \noindent Dla zadanych par wartości: $c$ oraz $x_0$ w podwójnej precyzji.

    \subsection*{Rozwiązanie}
        Program z rozwiązaniem: \texttt{ex6.jl}

    \subsection*{Wyniki i obserwacje}
        Zauważmy, że pukty stałe badanej funkcji możemy wyliczyć, rozwiązując równanie $\alpha = \alpha^2 + c$.
        \newline\newline
        Zatem dla $c = -2$ otrzymujemy punkty stałe $\alpha \in \{-1, 2\}$. Możemy więc na podstawie poniższych wykresów stwierdzić, że badana metoda jest zbieżna dla $x_0 \in \{1, 2\}$, natomiast dla $x_0 = 1.99999999999999$ metoda jest rozbieżna.
        \newline\newline
        Tak samo możemy wyznaczyć punkty stałe metody dla $c = -1$. Tutaj otrzymujemy wartości $\alpha = \frac{1 \pm \sqrt{5}}{2}$ - są to wartości niewymierne, więc nawet przy doborze odpowiednich danych ($x_0 = \alpha$) nie otrzymalibyśmy w praktyce ciągów zbieżnych przez wzgląd na błąd reprezentacji wartości $\alpha$.
        \newline
        Dla badanych całkowitych wartości $x_0$ ($x_0 \in \{1, -1\}$) otrzymujemy takie same ciągi (z dokładnością do pierwszego wyrazu, którego znak redukuje się po podnoszeniu do kwadratu), które zwracają cyklicznie wartości $0 = (\pm 1)^2 - 1$ oraz $-1 = 0^2 - 1$.
        \newline
        Natomiast dla $x_0 \in \{0.25, 0.75\}$ w pewnym momencie również otrzymujemy ciągi cykliczne $\{0, -1\}$, coSSS widać w \textit{tabeli 11}, jednak tutaj wynika to z błędów obliczeniowych przy podnoszeniu coraz mniejszych wartości do kwadratu (bardzo szybkie zmiejszanie wartości cechy) oraz odejmowania $c = 1$.

        \newpage

        \begin{figure}[h!]
            \centering
            \includegraphics[width=0.8\linewidth]{img/ex6_1.png}
            \label{fig:ex6_1}
        \end{figure}
        \begin{figure}[h!]
            \centering
            \includegraphics[width=0.8\linewidth]{img/ex6_2.png}
            \label{fig:ex6_2}
        \end{figure}
        \begin{figure}[h!]
            \centering
            \includegraphics[width=0.8\linewidth]{img/ex6_3.png}
            \label{fig:ex6_3}
        \end{figure}
        \begin{figure}[h!]
            \centering
            \includegraphics[width=0.8\linewidth]{img/ex6_4.png}
            \label{fig:ex6_4}
        \end{figure}
        \begin{figure}[h!]
            \centering
            \includegraphics[width=0.8\linewidth]{img/ex6_5.png}
            \label{fig:ex6_5}
        \end{figure}
        \begin{figure}[h!]
            \centering
            \includegraphics[width=0.8\linewidth]{img/ex6_6.png}
            \label{fig:ex6_6}
        \end{figure}
        \begin{figure}[h!]
            \centering
            \includegraphics[width=0.8\linewidth]{img/ex6_7.png}
            \label{fig:ex6_7}
        \end{figure}

        \begin{table}[h!]
        \centering
        \begin{tabularx}{0.55\textwidth}{l l l l}
            \hline
            $n$ & $x_0 = 1.0$ & $x_0 = 2.0$ & $x_0 = 1.99999999999999$ \\
            \hline
            $0$ & $1.0$ & $2.0$ & $1.99999999999999$ \\
            $1$ & $-1.0$ & $2.0$ & $1.99999999999996$ \\
            $2$ & $-1.0$ & $2.0$ & $1.9999999999998401$ \\
            $3$ & $-1.0$ & $2.0$ & $1.9999999999993605$ \\
            $4$ & $-1.0$ & $2.0$ & $1.999999999997442$ \\
            $5$ & $-1.0$ & $2.0$ & $1.9999999999897682$ \\
            $6$ & $-1.0$ & $2.0$ & $1.9999999999590727$ \\
            $7$ & $-1.0$ & $2.0$ & $1.999999999836291$ \\
            $8$ & $-1.0$ & $2.0$ & $1.9999999993451638$ \\
            $9$ & $-1.0$ & $2.0$ & $1.9999999973806553$ \\
            $10$ & $-1.0$ & $2.0$ & $1.999999989522621$ \\
            $11$ & $-1.0$ & $2.0$ & $1.9999999580904841$ \\
            $12$ & $-1.0$ & $2.0$ & $1.9999998323619383$ \\
            $13$ & $-1.0$ & $2.0$ & $1.9999993294477814$ \\
            $14$ & $-1.0$ & $2.0$ & $1.9999973177915749$ \\
            $15$ & $-1.0$ & $2.0$ & $1.9999892711734937$ \\
            $16$ & $-1.0$ & $2.0$ & $1.9999570848090826$ \\
            $17$ & $-1.0$ & $2.0$ & $1.999828341078044$ \\
            $18$ & $-1.0$ & $2.0$ & $1.9993133937789613$ \\
            $19$ & $-1.0$ & $2.0$ & $1.9972540465439481$ \\
            $20$ & $-1.0$ & $2.0$ & $1.9890237264361752$ \\
            $21$ & $-1.0$ & $2.0$ & $1.9562153843260486$ \\
            $22$ & $-1.0$ & $2.0$ & $1.82677862987391$ \\
            $23$ & $-1.0$ & $2.0$ & $1.3371201625639997$ \\
            $24$ & $-1.0$ & $2.0$ & $-0.21210967086482313$ \\
            $25$ & $-1.0$ & $2.0$ & $-1.9550094875256163$ \\
            $26$ & $-1.0$ & $2.0$ & $1.822062096315173$ \\
            $27$ & $-1.0$ & $2.0$ & $1.319910282828443$ \\
            $28$ & $-1.0$ & $2.0$ & $-0.2578368452837396$ \\
            $29$ & $-1.0$ & $2.0$ & $-1.9335201612141288$\\
            $30$ & $-1.0$ & $2.0$ & $1.7385002138215109$ \\
            $31$ & $-1.0$ & $2.0$ & $1.0223829934574389$ \\
            $32$ & $-1.0$ & $2.0$ & $-0.9547330146890065$ \\
            $33$ & $-1.0$ & $2.0$ & $-1.0884848706628412$ \\
            $34$ & $-1.0$ & $2.0$ & $-0.8152006863380978$ \\
            $35$ & $-1.0$ & $2.0$ & $-1.3354478409938944$ \\
            $36$ & $-1.0$ & $2.0$ & $-0.21657906398474625$ \\
            $37$ & $-1.0$ & $2.0$ & $-1.953093509043491$ \\
            $38$ & $-1.0$ & $2.0$ & $1.8145742550678174$ \\
            $39$ & $-1.0$ & $2.0$ & $1.2926797271549244$ \\
            $40$ & $-1.0$ & $2.0$ & $-0.3289791230026702$ \\
            \hline
        \end{tabularx}
        \label{table:ex6_c_2}
        \caption{Wartości wywołań funkcji rekurencyjnej w poszczególnych iteracjach dla $c = -2$}
    \end{table}

    \begin{table}[h!]
        \centering
        \begin{tabularx}{0.9\textwidth}{l l l l l}
            \hline
            $n$ & $x_0 = 1.0$ & $x_0 = -1.0$ & $x_0 = 0.75$ & $x_0 = 0.25$ \\
            \hline
            $0$ & $1.0$ & $-1.0$ & $0.75$ & $0.25$ \\
            $1$ & $0.0$ & $0.0$ & $-0.4375$ & $-0.9375$ \\
            $2$ & $-1.0$ & $-1.0$ & $-0.80859375$ & $-0.12109375$ \\
            $3$ & $0.0$ & $0.0$ & $-0.3461761474609375$ & $-0.9853363037109375$ \\
            $4$ & $-1.0$ & $-1.0$ & $-0.8801620749291033$ & $-0.029112368589267135$ \\
            $5$ & $0.0$ & $0.0$ & $-0.2253147218564956$ & $-0.9991524699951226$ \\
            $6$ & $-1.0$ & $-1.0$ & $-0.9492332761147301$ & $-0.0016943417026455965$ \\
            $7$ & $0.0$ & $0.0$ & $-0.0989561875164966$ & $-0.9999971292061947$ \\
            $8$ & $-1.0$ & $-1.0$ & $-0.9902076729521999$ & $-5.741579369278327e-6$ \\
            $9$ & $0.0$ & $0.0$ & $-0.01948876442658909$ & $-0.9999999999670343$ \\
            $10$ & $-1.0$ & $-1.0$ & $-0.999620188061125$ & $-6.593148249578462e-11$ \\
            $11$ & $0.0$ & $0.0$ & $-0.0007594796206411569$ & $-1.0$ \\
            $12$ & $-1.0$ & $-1.0$ & $-0.9999994231907058$ & $0.0$ \\
            $13$ & $0.0$ & $0.0$ & $-1.1536182557003727e-6$ & $-1.0$ \\
            $14$ & $-1.0$ & $-1.0$ & $-0.9999999999986692$ & $0.0$ \\
            $15$ & $0.0$ & $0.0$ & $-2.6616486792363503e-12$ & $-1.0$ \\
            $16$ & $-1.0$ & $-1.0$ & $-1.0$ & $0.0$ \\
            $17$ & $0.0$ & $0.0$ & $0.0$ & $-1.0$ \\
            $18$ & $-1.0$ & $-1.0$ & $-1.0$ & $0.0$ \\
            $19$ & $0.0$ & $0.0$ & $0.0$ & $-1.0$ \\
            $20$ & $-1.0$ & $-1.0$ & $-1.0$ & $0.0$ \\
            $21$ & $0.0$ & $0.0$ & $0.0$ & $-1.0$ \\
            $22$ & $-1.0$ & $-1.0$ & $-1.0$ & $0.0$ \\
            $23$ & $0.0$ & $0.0$ & $0.0$ & $-1.0$ \\
            $24$ & $-1.0$ & $-1.0$ & $-1.0$ & $0.0$ \\
            $25$ & $0.0$ & $0.0$ & $0.0$ & $-1.0$ \\
            $26$ & $-1.0$ & $-1.0$ & $-1.0$ & $0.0$ \\
            $27$ & $0.0$ & $0.0$ & $0.0$ & $-1.0$ \\
            $28$ & $-1.0$ & $-1.0$ & $-1.0$ & $0.0$ \\
            $29$ & $0.0$ & $0.0$ & $0.0$ & $-1.0$ \\
            $30$ & $-1.0$ & $-1.0$ & $-1.0$ & $0.0$ \\
            $31$ & $0.0$ & $0.0$ & $0.0$ & $-1.0$ \\
            $32$ & $-1.0$ & $-1.0$ & $-1.0$ & $0.0$ \\
            $33$ & $0.0$ & $0.0$ & $0.0$ & $-1.0$ \\
            $34$ & $-1.0$ & $-1.0$ & $-1.0$ & $0.0$ \\
            $35$ & $0.0$ & $0.0$ & $0.0$ & $-1.0$ \\
            $36$ & $-1.0$ & $-1.0$ & $-1.0$ & $0.0$ \\
            $37$ & $0.0$ & $0.0$ & $0.0$ & $-1.0$ \\
            $38$ & $-1.0$ & $-1.0$ & $-1.0$ & $0.0$ \\
            $39$ & $0.0$ & $0.0$ & $0.0$ & $-1.0$ \\
            $40$ & $-1.0$ & $-1.0$ & $-1.0$ & $0.0$ \\
            \hline
        \end{tabularx}
        \label{table:ex6_c_1}
        \caption{Wartości wywołań funkcji rekurencyjnej w poszczególnych iteracjach dla $c = -1$}
    \end{table}

\newpage

\end{document}
